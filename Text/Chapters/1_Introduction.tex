\chapter{Introduction} \label{chap:introduction}
% TEXT ==========================================

Stellar surface gravity is a fundamental property of all stars. The most direct means to determine stellar surface gravity is via measuring a stellar parallax combined with an apparent brightness and stellar effective temperature estimate \citep{GAIA,Prusti2016,Brown2016,Lindegren2016,Stassun2016}. In the absence of a stellar parallax, there are multiple techniques astronomers have developed to estimate stellar surface gravity including reduced proper motion  \citep{Lepine2011,Gould2003}, obtaining a high-resolution spectrum to characterize gravity sensitive absorption lines, measuring the photometric flicker with a space-based photometer \citep{Bastien2013,Bastien2016}, utilizing astro-density profiling for transiting planets \citep{Kipping2014}, direct intereferometric radii measurements \citep{Boyajian2012}, or via asteroseismic spectroscopic and photometric measurements \citep{Huber2014}. With the exception of reduced proper motion diagrams, these approaches are not readily available en masse for large-scale, all-sky surveys beyond the sensitivity limits of the GAIA mission \citep{GAIA}.

For the NASA \textit{Kepler} mission launched in 2009, the reliability of estimated stellar radii impacted the precision of the exoplanet parameters. Sub-giants masqueraded as main-sequence stars, leading to the underestimation of exoplanet radii \citep{Plavchan2014,Kane2014,Howell2014,Bastien2014,Mann2012}. This had secondary impacts on population-level science from \textit{Kepler} ranging from our understanding of planet formation mechanisms \citep{Lopez2013,Bodenheimer2014,Lopez2014,Schlichting2014,Rogers2015,Schlaufman2015}, to estimated occurrence rates and specifically $\eta_\oplus$, the occurrence of Earth-sized planets orbiting Sun-like stars in Habitable Zone orbits, as a function of planet radius and semi-major axis \citep{Everett2013, Plavchan2014, Zink2019, Zink2019_eta}. The \textit{GAIA} DR2 parallaxes reduced the typical random uncertainties in the estimated stellar radii to $\sim$4\%, and also reduced systematic uncertainties, increasing the radii of one-third of Kepler exoplanet host stars \citep{Bryson2020,Hsu2019,Berger2018}.  The \textit{GAIA} DR2 parallaxes also brought clarity to the astrophysical significance of the Fulton gap, demonstrating that super--Earths and mini--Neptunes are two distinct classes of planets \citep{Fulton2016,Fulton2017}. 

The NASA \textit{TESS} mission, launched in 2018, has surveyed most of the sky for the past two years to search for the nearest, brightest stars with transiting exoplanets \citep{Ricker2015}. The \textit{TESS} mission utilized a mix of photometry, proper motions and $\sim$250,000 parallaxes from Data Release 1 (hereafter:DR1) of the \textit{GAIA} mission to select before launch approximately $\sim$500,000 stars for two-minute cadence monitoring, constituting version 7 of the \textit{TESS} transit Candidate Target List (hereafter:tCTLv7), a subset of the full TESS Input Catalog (hereafter:TIC) \citep{Stassun2018}. Version 8 of the TIC and tCTL (hereafter:tCTLv8) were rebuilt to include \textit{GAIA} Data Release 2 (hereafter:DR2) \citep{DR2} which included over 1 billion parallax measurements, more than a thousand-fold increase of DR1. For \textit{TESS}, \textit{GAIA DR2} parallaxes allowed for the removal of subgiant and giants from tCTLv8 \citep{Stassun2019}\footnote{\url{https://filtergraph.com/tess_ctl}}. DR2 included parallaxes for most of the \textit{TESS} mission two minute cadence targets \citep{Stassun2019}, where many of the early \textit{TESS} mission discoveries have derived \citep{Barclay2018,pi_men,Newton2019,Nielsen2019,Plavchan2020}. However, a significant fraction of stars in the \textit{TESS} full frame images, including 24\% of the full TIC version 8 \citep{Stassun2019}, do not have \textit{GAIA} DR2 parallaxes and thus still have poorly constrained surface gravities. Obtaining prior estimates of the FFI field star surface gravities will impact the mission exoplanet yield, as $>$90\% of exoplanet candidates identified by the \textit{TESS} mission will be found around fainter targets in the full-frame images \citep{Barclay2018}.

In this work, we investigate whether colors from the \textit{2MASS} and \textit{WISE} photometric catalogs can provide additional surface gravity information. 2MASS revealed that photometric colors alone can sometimes be used as a proxy for the surface gravity of M stars, where M dwarfs and giants occupy bifurcated tracks in $J-H$ vs. $H-K_s$ \citep[Roc Cutri, 
priv. comm.,][]{Bessell1988,Carpenter2001,Ciardi2011,Plavchan2008,Plavchan2006,Skrutskie2006}. This difference was attributed to gravity-sensitive spectral features in the J-band \citep[]{Bessell1988}. We investigate whether colors from \textit{WISE} photometric catalogs can provide additional surface gravity information, extending the color separation for stars from the near-infrared into the mid-infrared. In \autoref{chap:2}, we present our sample of stars and how we filtered spectral and photometric data for color analysis. We use the Michigan Spectral Atlases for a compilation of spectroscopically determined spectral types and luminosity classes, which we compare to \textit{2MASS} and \textit{WISE} colors, and present our methodologies to determine color relationships as a function of spectral type and luminosity class.  In \autoref{chap:3}, we present the results of color analysis and an identification of a new color separation between dwarfs and giants for GK--types for $J-W_{1}$ and  $J-W_{2}$. In \autoref{chap:4}, we discuss the prospects of color selection for surface gravity in scope of modern astronomical surveys, and highlight a case study of assigning a probabilistic luminosity class by fitting color data to a probability density distribution. In \autoref{chap:conclusion}, we present our conclusions and suggestions for future improvement in luminosity class determination.

% FIGURES ==========================================



