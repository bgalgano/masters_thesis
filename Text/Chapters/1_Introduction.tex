\chapter{Introduction} \label{chap:introduction}
% TEXT ==========================================

The TESS mission will survey the entire sky for the nearest, brightest stars with transiting exoplanets \citep{Ricker2015,Sullivan2015}.  As we have learned from the Kepler mission, the reliability of estimated exoplanet radii is impacted by the precision of the stellar parameters \citep{Plavchan2014,Kane2014,Howell2014,Bastien2014,Mann2012}. In particular, sub-giants can masquerade as main-sequence stars in the absence of an ancillary determination of the stellar distance, leading to the underestimation of exoplanet radii. GAIA  provided the distances to the nearest $\sim$1 billion stars, enabling the accurate determination of those stellar radii \citep{Prusti2016,Brown2016,Lindegren2016,Stassun2016}

% The TESS mission utilized a mix of photometry, proper motions and GAIA DR1 parallaxes to select approximately half a million stars for high cadence monitoring; the initial version 7 TESS transit Candidate Target List or tCTL), many of which now have GAIA DR2 parallaxes as compiled in the revised version 8 tCTL. However, many more stars in the TESS full frame images do not have GAIA DR2 parallaxes and thus have unconstrained surface gravities. Obtaining prior estimates of the FFI field star surface gravities will impact the mission exoplanet yield, as $>$90\% of exoplanet candidates identified by the TESS mission will be found around fainter targets in the full-frame images.

In the absence of a stellar parallax, several tools exist to determine the surface gravity of stars.  Flicker and asteroseismology require high precision photometry, which won't be available until after the launch of the TESS mission \citep{Bastien2013,Bastien2016}. Prior to launch, obtaining high resolution spectroscopy in a short time-frame is not practical for hundreds of thousands of stars dispersed across the sky.  Thus, the combination of broad-band photometry and reduced proper-motions as a proxy for distance are utilized to identify main sequence stars \citep{Bastien2016,Huber2014,Lepine2011,Brown2011,Gould2003}. 2MASS revealed that photometric colors alone can sometimes be used as a proxy for the surface gravity of M stars, where M dwarfs and giants occupy slightly different tracks in $J-H$ vs. $H-K_s$ \citep{Ciardi2011,Plavchan2008,Plavchan2006,Skrutskie2006}.  It is not known whether this difference is attributable to extinction or gravity-sensitive spectral features in the J-band.  

In this work, we investigate whether colors from the 2MASS and WISE photometric catalogs can provide additional surface gravity information, extending the separation for stars into the mid-infrared. In \autoref{chap:2}, we present our sample of stars and how we filtered spectral and photometric data for color analysis. We use the Michigan Spectral Atlases for a compilation of spectroscopically determined spectral types and luminosity classes, which we compare to 2MASS and WISE colors, and present our methodologies to determine color relationships as a function of spectral type and luminosity class.  In \autoref{chap:3}, we present the results of color analysis and an identification of a new color separation between dwarfs and giants for GK-types for $J-W_{1}$ and  $J-W_{2}$. In \autoref{chap:4}, we discuss the prospects of color selection for surface gravity in scope of modern astronomical surveys, and highlight a case study of assigning a probabilistic luminosity class by fitting color data to a probability density distribution. In \autoref{chap:conclusion}, we present our conclusions and suggestions for future improvement in luminosity class determination.

% FIGURES ==========================================



