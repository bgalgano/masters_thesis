\chapter{Conclusion}\label{chap:conclusion}
% TEXT ==========================================

We have discovered a color separation between G-type and early K-type (G0--K3) dwarf and giant stars that is statistically significant for $J-W_{1}$ and $J-W_{2}$. Giant stars (class $IV$ or below) are separated from dwarf stars (class $V$), on average, by $\sim$0.10 mag. for $J-W_{1}$ and $\sim$0.14 mag. for $J-W_{2}$ for these spectral sub-types. We can use this color separation to estimate luminosity class likelihood just from magnitude of the infrared, with dereddening correction, as long as the star is estimated to be a G or early K star. We can better constrain this luminosity class estimate by incorporating an artificial galactic population model with these infrared color magnitudes (further discussed in Appendix \ref{Appendix:4}.

The range and accuracy this luminosity class estimator tool is topped off due to the limited sample size of the Michigan Spectral Atlas Catalogs \citep{Houk1975,Houk1978,Houk1982,Houk1988,Houk1999} cross-matched to infrared catalogs 2MASS \citep{2MASS} and WISE \citep{WISE}. Few spectra are available in the atlas for M-types and very early types (OBA) because of the difficulty in obtaining decent quality observations for low luminosity targets and the rarity in stellar populations, respectively. Future improvements could include expanding the color-luminosity class estimation tool to these other spectral types by having bigger sample sizes. In order to do this, we would need obtain targets whose luminosity class and spectral type we can obtain with a high degree of certainty, combine this data with infrared photometry, and then calibrate for the color difference in a similar fashion as we did for Michigan stars in this work. However, it is a possibility that a color difference between dwarfs and giants may not exist for these other spectral sub-types or for colors. In this case, one can explore different colors or even color-color combinations to see if a separation between dwarfs and giants do exist for various other parameter spaces, by adjusting spectral subtype, color, or luminosity class as the experimental variables. Regardless, this poses potential to be a much cheaper and efficient method of characterizing stars in bulk with only photometry and a spectral type estimate, rather than high-cost methods mentioned in Chapter \ref{chap:introduction}. This color separation has the potential to be applied to current/upcoming large-scale astronomical surveys, and can be an important tool to constrain exoplanet radii by constraining the size of their host stars.

% FIGURES ==========================================
