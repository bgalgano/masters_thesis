\chapter{Conclusion}\label{chap:conclusion}
% TEXT ==========================================

We have discovered a color separation between G-type and early K-type dwarf and giant stars that is statistically significant for \jwone and \jwtwo; giants stars (class $IV$ or below) are on average [filler stat] more red than their dwarf counterparts (class $V$) for the same spectral type. We can use this color separation to estimate luminosity class likelihood just from magnitude of the infrared color without needing extinction correction, as long as the star is estimated to be a G-type or early K-type. We can better constrain this luminosity class estimate by incorporating an artificial galactic population model with these infrared color magnitudes by a further [filler stat].

This luminosity class and color probability could also be incorporated into a Bayesian framework using other empirical measurements to improve the fidelity of surface gravity estimates for GKM stars in the absence of parallaxes, high resolution spectra, flicker analysis, or asteroseismology. 

The range and accuracy this luminosity class estimator tool is topped off due to the limited sample size of the Michigan Spectral Atlas Catalogs \citep{Houk1975,Houk1978,Houk1982,Houk1988,Houk1999} cross-matched to infrared catalogs 2MASS \citep{2MASS} and WISE \citep{WISE}. Few spectra are available in the atlas for M-types and very early types (OBA) because of the difficulty in obtaining decent quality observations for low luminosity targets and the rarity in stellar populations, respectively. Future improvements could include expanding the color-luminosity class estimation tool to these other spectral types. In order to do this, we would need obtain targets whose luminosity class and spectral type we can obtain with a high degree of certainty, and combined  infrared photometry, calibrate for the color difference in a similar fashion as we did for the G/K-types in this work. However, it is a possibility that a color difference between dwarfs and giants may not exist for these other spectral sub-types or for colors. In this case, one can explore different colors or even color-color combinations to see if a separation between dwarfs and giants do exist for various other parameter spaces, by adjusting spectral subtype, color, or luminosity class as experimental variables. Regardless, this poses potential to be a cheaper and efficient method of characterizing stars in bulk with only photometry and a spectral type estimate. It will be useful in the scope of future large-scale surveys and exoplanet characterization by constraining host stars

% FIGURES ==========================================
