\phantomsection
\addcontentsline{toc}{chapter}{Abstract}
\vspace*{\fill}
\begin{center}
\textbf{Abstract}
% ==========================================

It is challenging to distinguish nearby field dwarfs and distant evolved stars of the same spectral type without a parallax, high resolution spectrum, flicker, or asteroseismic measurements. Determining the surface gravity of a star is particularly important for finding small transiting planets around bright, nearby main sequence stars. The TESS mission has utilized a mix of photometry, proper motions and GAIA parallaxes to select approximately half a million stars for high cadence monitoring (the TESS transit Candidate Target List or tCTL). Knowing a priori estimate of the target surface gravities will impact the mission exoplanet yield.  In this study, we are motivated by the separation of M giants and M dwarfs in the 2MASS photometric $J-H$ colors to explore whether or not this separation extends further into the infrared with the WISE survey. We present 2MASS and WISE colors of \bincount stars from the Michigan Spectral Atlases, and find a new separation between GK dwarfs and giants in $J-W{1}$ and $J-W{2}$, with an average separation of $\sim$0.10 and $\sim$0.14 magnitude, respectively. We found no strong dependence between color separation and galactic latitude ($b$), which suggests this separation is not due to an observational bias from extinction but rather an astrophysical process. We note that the separation is subtle and is at the border of 2MASS/WISE average certainty ($\sim$0.100-0.15 mag.), and that this separation disappears for stars fainter than TESS magnitude $\sim$15 due to reddening effects. In the absence of other measures of surface gravity, these colors may be used to estimate a probabilistic luminosity class and act as a low--cost, efficient stellar characterization tool.

\end{center}
\vspace{\fill}
