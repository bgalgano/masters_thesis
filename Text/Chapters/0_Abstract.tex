\phantomsection
\addcontentsline{toc}{chapter}{Abstract}
\vspace*{\fill}
\begin{center}
\textbf{Abstract}
% ==========================================

The surface gravity of a star is particularly important for distinguishing transiting exoplanets orbiting main sequence stars from more distance evolved star eclipsing binaries and blends. While the NASA \textit{TESS} mission has measured \textit{GAIA DR2} parallaxes for most of its two-minute cadence targets, many potential exoplanet hosts in the full-frame images will not have parallax measurements. In this study, we are motivated by the color bifurcation of M giants and M dwarfs in the \textit{2MASS} photometric $J-H$ colors to explore whether or not a similar color separation also exists further into the infrared with the \textit{WISE} survey. We present \textit{2MASS} and \textit{WISE} colors of \bincount stars from the Michigan Spectral Atlases, and find a new separation specifically between G \& K spectral type dwarfs and giants in $J-W_{1}$ and $J-W_{2}$, with an average separation of $\sim$0.10 and $\sim$0.14 magnitude, respectively. We found no strong dependence between color separation and galactic latitude ($b$), which combined with the spectral type dependence suggests this separation is not due to an observational bias from extinction but rather an astrophysical effect. The separation is subtle and is comparable to the typical \textit{2MASS} and \textit{WISE} color uncertainty in our sample ($\sim$0.10--0.15 mag.). When we extend our work from the Michigan Spectral Atlas to the full \textit{TESS} catalog, this separation in color is obscured for stars fainter than \textit{TESS} magnitude $\sim$10 due to reddening effects. In the absence of other measures of surface gravity such as a parallax, these colors, when dereddened, may be used to estimate a probabilistic luminosity class and act as a low--cost, efficient tool that can be utilized for stellar and exoplanet characterization.

\end{center}
\vspace{\fill}
