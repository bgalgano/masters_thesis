\phantomsection
\addcontentsline{toc}{chapter}{Abstract}

\vspace*{\fill}
\begin{center}

\textbf{Abstract}
% ========================

It is challenging to distinguish nearby field dwarfs and distant evolved stars of the same spectral type without a parallax, high resolution spectrum, flicker, or asteroseismic measurements. Determining the surface gravity of a star is particularly important for finding small transiting planets around bright, nearby main sequence stars. The TESS mission will utilize a mix of photometry, proper motions and GAIA parallaxes to select approximately half a million stars for high cadence monitoring (the TESS transit Candidate Target List or tCTL). Knowing a priori an estimate of the target surface gravities will impact the mission exoplanet yield.  In this study, we are motivated by the separation of M giants and M dwarfs in the 2MASS photometric J-H colors to explore whether or not this separation extends further into the infrared with the WISE survey.  We present 2MASS and WISE colors of stars from the Michigan Spectral Atlases, and find a new separation between GK dwarfs and giants in \jwone and \jwtwo.  In the absence of other measures of surface gravity, this color may be used to estimate a probabilistic luminosity class.  This may be combined with a Galactic population model to optimize TESS mission target selection.


\end{center}
\vspace{\fill}
