\phantomsection
\addcontentsline{toc}{chapter}{Abstract}
\vspace*{\fill}
\begin{center}
\textbf{Abstract}
% ==========================================

The surface gravity of a star is particularly important for distinguishing transiting exoplanets orbiting main sequence stars from more distance evolved star eclipsing binaries and blends. There are multiple techniques to estimate stellar surface gravity, including the most direct of measuring a parallax, obtaining a high-resolution spectrum, measuring the photometric flicker with a space-based photometer, utilizing astro-density profiling for transiting planets, or via asteroseismic spectroscopic and photometric measurements. However, these approaches are not readily available en masse for large-scale surveys beyond the sensitivity limits of the GAIA mission. While the NASA TESS mission has measured GAIA DR2 parallaxes for most of its two-minute cadence targets, many potential exoplanet hosts in the full-frame images will not.  In this study, we are motivated by the separation of M giants and M dwarfs in the 2MASS photometric $J-H$ colors to explore whether or not this separation extends further into the infrared with the WISE survey. We present 2MASS and WISE colors of \bincount stars from the Michigan Spectral Atlases, and find a new separation between GK dwarfs and giants in $J-W_{1}$ and $J-W_{2}$, with an average separation of $\sim$0.10 and $\sim$0.14 magnitude, respectively. We found no strong dependence between color separation and galactic latitude ($b$), which suggests this separation is not due to an observational bias from extinction but rather an astrophysical effect. The separation is subtle and is comparable to the typical 2MASS/WISE color uncertainty in our sample ($\sim$0.10--0.15 mag.). When we extend our work from the Michigan Spectral Atlas to the TESS catalog, this separation in color is obscured for stars fainter than TESS magnitude $\sim$10 due to reddening effects. In the absence of other measures of surface gravity such as a parallax, these colors, when dereddened, may be used to estimate a probabilistic luminosity class and act as a low--cost, efficient tool that can be utilized for stellar and exoplanet characterization.

\end{center}
\vspace{\fill}
