\chapter{Spectral and infrared photometry catalogs}\label{chap:DATA}
% ==========================================

We used archival information to construct our sample of spectroscopic and infrared photometric measurements of nearby stars, and we discuss each in turn.

\section{Michigan Spectral Atlases} \label{sec:michigan}

We utilize the five volumes of the Michigan Spectral atlases \citep[]{Houk1975, Houk1978, Houk1982,Houk1988,Houk1999}. These catalogs characterize 150,952 HD stars over a range in B1900 declinations (Table 1) for HD numbers $<$225300 (e.g., not including extensions to the HD catalog) with photographic magnitudes as faint as $\sim$9.

The spectra were exposed on objective-prism plates at the Michigan Curtis Schmidt Telescope at Cerro Tololo Inter-American Observatory. The spectra, with an average resolution of $\sim$2$\circ/mm$, were classified visually in spectral type and luminosity class (e.g. ``G0IV'', ``K2V'', etc.).  

\section{2MASS and WISE Photometry} \label{subsec:2MASS_WISE}

All entries of the Michigan spectral atlas of HD stars were cross-matched with 2MASS \citep{Skrutskie2006} by using the SIMBAD web tool\footnote{Available at \url{http://simbad.u-strasbg.fr}} to obtain their 2MASS designations. A cross-match to AllWISE \cite{ALLWISE,ALLWISE-dwarfs} was then performed with the method described by \cite{BANYAN}. In summary, AllWISE entries within 3\arcsec of the 2MASS position were already identified in the AllWISE catalog available at the NASA/IPAC Science Archive\footnote{Available at \url{http://irsa.ipac.caltech.edu/}}. Cross-matches at further separations were performed by identifying the AllWISE entry nearest to each 2MASS position, and verifying that no other 2MASS entry is closer to the resulting AllWISE position.

As a result of our positional cross-matching vetting and photometric quality criteria, not all stars in our sample have photometric measurements in all bands, and the counts vary from color-to-color.  For a given color, there are on the order of $\sim$20,000 stars from the Michigan Spectral Atlas with good position cross-matches and photometry from 2MASS and WISE.




