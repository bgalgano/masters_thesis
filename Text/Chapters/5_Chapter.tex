\chapter{Using color separation as a stellar characterization tool}\label{chap:DISCUSSION}
% ==========================================
\section{Color probability density functions}
We have considered two possibilities for this observable difference in the infrared colors of dwarf stars compared to giant stars. We have considered that the source of the color difference is due to (1) observational bias from reddening due to galactic extinction, or is (2) an astrophysical phenomenon that helps underline the intrinsic separation of dwarfs from giant stars. We stipulate that while there is a possibility both (1) and (2) could be contributing to the color difference we do see among different luminosity classes, it is more likely that (2) is the more dominant effect, which has interesting astrophysical implications in characterizing dwarfs from giants stars. We have examined the medians magnitudes of colors \jwone and \jwtwo across the galactic latitude coordinate $b$, and saw no strong dependence for $-15<b<15$ where extinction is greatest, and may have contributed more reddened colors that could have explained this difference (Figure \ref{fig:color-vs-b}).


\section{Applicability to modern surveys and range of use}


Therefore J is just damn special, and it would be a subject of future work as to why J band spectral features are gravity sensitive for GKM dwarfs (J-K for M dwarfs) and J-W1/2 for GK dwarfs.

We also note the importance the J-band is in differentiating between dwarfs from giants given its previously proved sensitivity for differentiating M dwarfs from GK dwarfs for J-K (REF). The combination of this study with J-W1/2 in being sensitive for GK dwarfs, and J-K being sensitive for GKM stars suggests the presence of surface gravity sensitive lines in the J-band. Thus, there we suggest it is viable to search for gravity sensitive lines along these the J-band wavelength range.